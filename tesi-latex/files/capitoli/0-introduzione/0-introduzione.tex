\chapter{Introduzione}\label{ch:introduzione}

L'Object Detection è una delle aree più dinamiche ed importanti della Computer Vision, che mira ad individuare e classificare oggetti all'interno di un'immagine. Questa tecnologia trova applicazione in diversi settori, tra cui la sorveglianza, la guida autonoma e la robotica, dove è fondamentale riconoscere oggetti e persone in tempo reale e in condizioni variabili.

In particolare, le immagini termiche offrono un vantaggio significativo in situazioni di scarsa visibilità, come di notte o in condizioni atmosferiche avverse, permettendo di rilevare oggetti sulla base del calore che emettono. Tuttavia, lavorare con immagini termiche presenta delle sfide uniche, tra cui una minore risoluzione spaziale ed un minor contrasto rispetto alle immagini a luce visibile.

Questa tesi si propone di esplorare l'efficacia di diversi modelli di Object Detection applicati a immagini termiche e di migliorare le loro prestazioni attraverso l'uso di tecniche di Data Augmentation. Il lavoro sperimentale alla base di questa ricerca si concentra su tre tipologie di modelli YOLO: YOLOv8, YOLO-World e RT-DETR.

I primi tre capitoli della tesi forniscono una base teorica essenziale per comprendere le tecniche e le metodologie utilizzate, mentre il quarto capitolo è dedicato al lavoro sperimentale svolto con l'obiettivo di identificare il modello più efficiente nel riconoscimento di oggetti in immagini termiche.