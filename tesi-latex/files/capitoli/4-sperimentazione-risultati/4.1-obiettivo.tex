\section{Obiettivo}
L'obiettivo di questa tesi sperimentale è confrontare le prestazioni di tre modelli avanzati di Object Detection applicati alle immagini termiche: YOLOv8, YOLO-World e RT-DETR. Questi modelli sono stati scelti per rappresentare tre diversi approcci al problema del riconoscimento degli oggetti, consentendo così un confronto dettagliato delle loro capacità su questo specifico tipo di dati.

Per massimizzare le prestazioni di questi modelli, espanderemo il dataset di immagini termiche utilizzando tecniche di augmentation, valutando le più efficaci offerte dalla libreria Albumentations. Successivamente, testeremo i tre modelli in questi scenari per identificare quello che meglio si adatta al riconoscimento di oggetti in immagini termiche e, soprattutto, ottiene il miglioramento più significativo grazie all'addestramento su dataset aumentato.

\newpage
