\section{Applicazioni dell'Object Detection}

L'object detection è una tecnologia fondamentale con numerose applicazioni pratiche in vari settori, quali:
\begin{enumerate}
    \item \textbf{Sicurezza e Sorveglianza}: utilizzata per monitorare ambienti e individuare attività sospette. Esempi includono il rilevamento di persone, veicoli o oggetti non autorizzati in aree sensibili come aeroporti, banche e stazioni ferroviarie.
    \item \textbf{Automazione Industriale}: usata per il controllo di qualità, l'ispezione automatizzata dei prodotti e la gestione degli inventari.
    \item \textbf{Guida Autonoma e Veicoli Intelligenti}: consente ai veicoli di rilevare e reagire in tempo reale a pedoni, veicoli, segnaletica stradale e altri ostacoli.
    \item \textbf{Sanità e Medicina}: supporta la diagnostica medica automatizzata, il monitoraggio dei pazienti e la ricerca scientifica.
    \item \textbf{Commercio e Marketing}: impiegata per migliorare l'esperienza del cliente, ottimizzare la gestione del magazzino e analizzare il comportamento dei consumatori, tramite l'analisi delle immagini e dei video permette di monitorare gli scaffali nei negozi e di rilevare automaticamente il movimento dei prodotti.
    \item \textbf{Sicurezza Stradale e Controllo del Traffico}: fondamentale per il miglioramento della sicurezza stradale e l'ottimizzazione del controllo del traffico urbano, grazie a sistemi avanzati che possono rilevare incidenti, monitorare il flusso veicolare, riconoscere targhe di registrazione e applicare le normative stradali.
\end{enumerate}
